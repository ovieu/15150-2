\documentclass[11pt]{article}
\usepackage{enumerate}
\usepackage{graphicx}
\usepackage{fancyhdr}
\usepackage{amsmath, amsfonts, amsthm, amssymb}
\usepackage[margin=1in,headsep=.2in]{geometry}
\usepackage{tikz}
\usepackage{fancyvrb}

\setlength{\headheight}{13.6pt}
\setlength{\parindent}{0pt}
\setlength{\parskip}{5pt plus 1pt}

\def\indented#1{\list{}{}\item[]}
\let\indented=\endlist

\newcounter{questionCounter}
\newcounter{partCounter}[questionCounter]
\newenvironment{question}[2][\arabic{questionCounter}]{%
    \setcounter{partCounter}{0}%
    \vspace{.25in} \hrule \vspace{0.5em}%
        \noindent{\bf #2}%
    \vspace{0.8em} \hrule \vspace{.10in}%
    \addtocounter{questionCounter}{1}%
}{}
\renewenvironment{part}[1][\alph{partCounter}]{%
    \addtocounter{partCounter}{1}%
    \vspace{.10in}%
    \begin{indented}%
       {\bf (#1)} %
}{\end{indented}}

%%%%%%%%%%%%%%%%% Identifying Information %%%%%%%%%%%%%%%%%
%% This is here, so that you can make your homework look %%
%% pretty when you compile it.                           %%
%%         DO NOT PUT YOUR NAME ANYWHERE ELSE!!!!        %%
%%%%%%%%%%%%%%%%%%%%%%%%%%%%%%%%%%%%%%%%%%%%%%%%%%%%%%%%%%%
\newcommand{\myclass}{15-150}
\newcommand{\myname}{Jonathan Li}
\newcommand{\myandrew}{jlli}
\newcommand{\myhwname}{Assigment 2}
\newcommand{\myrecitation}{Section S}
%%%%%%%%%%%%%%%%%%%%%%%%%%%%%%%%%%%%%%%%%%%%%%%%%%%%%%%%%%%

\pagestyle{fancyplain}
\lhead{\fancyplain{}{\myname}}
\chead{\fancyplain{}{\textbf{\myclass{} \myhwname}}}
\rhead{\fancyplain{}{\myandrew}}

\begin{document}
\thispagestyle{plain}

\begin{center}
{\Large \myclass{} \myhwname} \\
\myname \\
\myandrew \\
\myrecitation \\
\today
\end{center}

\begin{question}{Task 2.1}
\underline{Theorem 1:} For all values \verb!l : (string * int) list!, \verb!zip(unzip l)! $\cong$ \verb!l!.\\
Proof: By structural induction on \verb!l!.\\
\textbf{Case \texttt{[]}:} To show: \verb!zip(unzip [])! $\cong$ \verb![]!\\
\underline{Proof:}
\begin{align*}
    \texttt{zip(unzip [])} & \cong && \texttt{zip(case [] of [] => ([],[])| }\dots\texttt{)} &&& \text{[step]}\\
    & \cong && \texttt{zip(([],[]))} &&& \text{[step]}\\
    & \cong && \texttt{case ([],[]) of ([],\char`_) => [] | }\dots\texttt{)} &&& \text{[step]}\\
    & \cong && \texttt{[]}
\end{align*}
By extensional equivalence, \verb!zip(unzip [])! $\cong$ \verb![]!.\\
\textbf{Case \texttt{(x1, x2)::xs}} for some \verb!x1!, \verb!x2!, \verb!xs!.\\
Inductive Hypothesis: \verb!zip(unzip xs) => xs!\\
\hspace{1cm} \textbf{*Note I:} \verb!unzip xs => (fst unzip xs, snd unzip xs)!\\
\hspace{1cm} \textbf{*Note II:} The IH $\implies$ \verb!unzip xs! is a valuable expression\\
\qquad To show: \verb!zip(unzip((x1, x2)::xs)! $\cong$ \verb!(x1, x2)::xs!\\
\underline{Proof:}
\begin{align*}
    \texttt{zip(unzip (x1, x2)::xs)} &\cong && \texttt{zip(case (x1, x2) of [] => ([],[]) | }\dots\texttt{)} &&& \text{[step]}\\
    &\cong && \texttt{zip(let val (l1, l2) = unzip xs in}\dots\texttt{)} &&& \text{[step]}\\
    &\cong && \texttt{zip(x1::fst(unzip xs), x2::snd(unzip xs))} &&& \text{[Rule 1,}\\
    & && &&& \text{Note I,}\\
    & && &&& \text{Note II]}\\
    &\cong && \texttt{case (x1::fst(unzip xs), x2::snd(unzip xs) of}\dots\texttt{)} &&& \text{[step]}\\
    &\cong && \texttt{(x1, x2)::zip(fst (unzip xs), snd (unzip xs))} &&& \text{[step]}\\
    &\cong && \texttt{(x1, x2:)::zip(unzip xs)} &&& \text{[Rule 2]}\\
    &\cong && \texttt{(x1,x2)::xs} &&& \text{[IH]}
\end{align*}
By extensional equivalence, \verb!zip(unzip (x1, x2)::xs)! $\cong$ \verb!(x1, x2)::xs!.\\
Since the Base Case and the Inductive Step hold, Theorem 1 must be true.
\end{question}

\newpage

\begin{question}{Task 2.2}
\underline{Theorem 2:} For all values \verb!l1 : string list!, \verb!l2 : int list!, \verb!unzip(zip(l1, l2))! $\cong$ \verb!(l1, l2)!\\
\textbf{Theorem 2 is false:} Proof by counterexample.\\
Let \verb!l1 = ["hi"]!, \verb!l2 = [1, 2]!. Then,
\begin{align*}
    \texttt{unzip(zip(l1, l2))} &\cong && \texttt{unzip(zip(["hi"], [1, 2]))} &&&\\
    &\cong && \texttt{unzip(case (["hi],[1, 2]) of}\dots\texttt{)} &&& \text{[step]}\\
    &\cong && \texttt{unzip(("hi",1)::zip([],[2]))} &&& \text{[step]}\\
    &\cong && \texttt{unzip(("hi",1)::[]} &&& \text{[step]}\\
    &\cong && \texttt{unzip([("hi",1)])} &&& \text{[step]}\\
    &\cong && \texttt{case [("hi",1)] of [] => ([],[])| }\dots\ &&& \text{[step]}\\
    &\cong && \texttt{let val (l1, l2) = unzip [] in ("hi"::l1, 1::l2) end} &&& \text{[step]}\\
    &\cong && \texttt{("hi"::fst(unzip []), 1::snd(unzip []))} &&& \text{[Rule 1]}\\
    &\cong && \texttt{("hi"::[], 1::[])} &&& \text{[step]}\\
    &\cong && \texttt{(["hi"],[1])} &&& \text{[step]}
\end{align*}
$\therefore$ \verb!unzip(zip(["hi"], [1, 2, 3]))! $\implies$ \verb!(["hi"],[1])! $\not\cong$ \verb!(["hi"], [1, 2, 3])!\\
Thus by counterexample, Theorem 2 is false.
\end{question}

\newpage

\begin{question}{Task 5.2}
Given a list \verb!L! with length $n$, the asymptotic bound for the work of \verb!prefixSum L! is $n^2$. In other words, $W_{\texttt{prefixSum}}(n)$ is $O(n^2)$, polynomial time.
\end{question}

\begin{question}{Task 5.4}
Given a list \verb!L! with length $n$, the asymptotic bound for the work of \verb!prefixSumFast L! is $n$. In other words, $W_{\texttt{prefixSumFast}}(n)$ is $O(n)$, linear time.
\end{question}

\end{document}{center}